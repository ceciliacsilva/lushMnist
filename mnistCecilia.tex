% Created 2016-06-04 Sáb 20:36
\documentclass[11pt]{article}
\usepackage[utf8]{inputenc}
\usepackage[T1]{fontenc}
\usepackage{fixltx2e}
\usepackage{graphicx}
\usepackage{longtable}
\usepackage{float}
\usepackage{wrapfig}
\usepackage{soul}
\usepackage{textcomp}
\usepackage{marvosym}
\usepackage{wasysym}
\usepackage{latexsym}
\usepackage{amssymb}
\usepackage{hyperref}
\tolerance=1000
\providecommand{\alert}[1]{\textbf{#1}}

\title{IA}
\author{Cecília Carneiro e Silva, Valks}
\date{}
\hypersetup{
  pdfkeywords={},
  pdfsubject={},
  pdfcreator={Emacs Org-mode version 7.9.3f}}

\begin{document}

\maketitle

\setcounter{tocdepth}{3}
\tableofcontents
\vspace*{1cm}

\section{Implementação}
\label{sec-1}


  Segue abaixo a implementação de uma rede neural em Lush \href{http://lush.sourceforge.net/}{http://lush.sourceforge.net/}, mais especificadamente uma `Convolutional Neural Networks' (ConvNets) . As diferenças entre elas se resume no fato das ConvNets assumirem claramente que desejam que sua estrada seja uma imagem, com isso, algumas propriedades são `escondidas'.

  O código abaixo executa o clássico problema MNIST, classificação de números manuscritos. Entraremos com imagens e seus respectivos labels, o treinamento será feito com 60000 <imagens, labels>, disponíveis em: \href{http://yann.lecun.com/exdb/mnist/}{http://yann.lecun.com/exdb/mnist/} .
  
  

\end{document}
